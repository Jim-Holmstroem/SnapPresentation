\documentclass{beamer}
\usepackage{minted}

\usetheme{Madrid}

\title{Snap}
\author{Jim Holmstr\"{o}m}
\date{\today}

\begin{document}
\begin{frame}
  \titlepage
\end{frame}

\section*{Outline}
\begin{frame}
  \tableofcontents
\end{frame}

\section{Introduction}
\subsection{Overview of Snap}

\section{Usage}
\subsection{usage?}


\begin{frame}[fragile]{}
  \frametitle{Fibonacci}
  \begin{minted}[]{haskell}
  -- Type annotation (optional)
  fib :: Int -> Integer

  -- With self-referencing data
  fib n = fibs !! n
    where fibs = 0 : scanl (+) 1 fibs
  \end{minted}
\end{frame}
\begin{frame}[fragile]{}
  \frametitle{Fibonacci cont.}
  \begin{minted}[]{haskell}
  -- Same, coded directly
  fib n = fibs !! n
    where fibs = 0 : 1 : next fibs
          next (a : t@(b:_)) = (a+b) : next t

  -- Similar idea, using zipWith
  fib n = fibs !! n
    where fibs = 0 : 1 : zipWith (+) fibs (tail fibs)

  -- Using a generator function
  fib n = fibs (0,1) !! n
    where fibs (a,b) = a : fibs (b,a+b)
  \end{minted}
\end{frame}
\begin{frame}
    ajs
\end{frame}

\end{document}
